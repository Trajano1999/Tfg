\addchap{Conclusiones}

    Durante el desarrollo de este trabajo, hemos realizado un análisis de los fundamentos teóricos y prácticos de algunos criptosistemas de clave pública, poniendo especial atención al criptosistema de Merkle-Hellman. Dado que este criptosistema se basa en el problema de la mochila, hemos abarcado una gran variedad de conceptos necesarios para su compresión, que van desde nociones como las máquinas de Turing, hasta cuestiones tan fundamentales como el problema P vs NP.

    A lo largo de este proyecto, además de explicar toda la teoría matemática apoyada de ejemplos esclarecedores, hemos destacado la relevancia práctica de los criptosistemas mediante su implementación. Esto nos ha permitido analizar los resultados obtenidos y compararlos con los teóricos esperados. De esta manera, hemos conseguido plasmar todo el conocimiento estudiado en aplicaciones concretas.

    Asimismo, hemos abordado de manera firme los ataques principales a los criptosistemas estudiados, entendiendo que resulta casi imposible garantizar la total seguridad de un sistema criptográfico. Incluso en la actualidad, persistimos en la búsqueda de dicho criptosistema, especialmente ante el desafío que representan los algoritmos cuánticos, los cuales ponen en riesgo a muchos de los criptosistemas actuales.
    
    Para finalizar, este trabajo de fin de grado ha supuesto el primer acercamiento a la actividad profesional, y como tal, este proyecto queda abierto ante futuras investigaciones y desarrollos en el campo de la criptografía y de la computación cuántica.
    
\endinput
