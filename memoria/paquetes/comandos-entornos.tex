% DEFINICIÓN DE COMANDOS Y ENTORNOS

% CONJUNTOS DE NÚMEROS
  \newcommand{\N}{\mathbb{N}}     % Naturales
  \newcommand{\R}{\mathbb{R}}     % Reales
  \newcommand{\Z}{\mathbb{Z}}     % Enteros
  \newcommand{\Q}{\mathbb{Q}}     % Racionales
  \newcommand{\C}{\mathbb{C}}     % Complejos

% Otros comandos propios
  \newcommand{\mun}{\{\mu_n\}_{n \in \N_0}}                  				        % Sucesión de momentos
  \newcommand{\pol}{\mathbb{P}[x]}                           				        % Polinomios
  \newcommand{\funl}{\mathcal{L}}                            				        % Funcional de momentos L
  \newcommand{\spol}{\{P_n\}_{n \in \N_0}}                   				        % Sucesión de polinomios
  \newcommand{\pescalar}[2]{\langle #1, #2 \rangle}      				            % Producto escalar
  \newcommand{\charlier}{\{C_n^{(\mu)}\}_{n \in \N_0}}       			   	      % SPOVD de Charlier
  \newcommand{\meixner}{\{M_n^{(\beta, c)}\}_{n \in \N_0}}                 	% SPOVD de Meixner
  \newcommand{\kraw}{\{K_n^{(p)}\}_{n \in \{0, \dots, N-1\}}}              	% SPOVD de Krawtchouk
  \newcommand{\hahn}{\{h_n^{(\alpha, \beta)}\}_{n \in \{0, \dots, N-1\}}}	  % SPOVD de Hahn
  \newcommand{\eprob}{(\Omega, \mathcal A, P)}              				        % Espacio de probabilidad
  \newcommand{\pest}{\{X_t\}_{t \in T}}                     				        % Proceso estocástico
  \newcommand{\markov}{\{X_t\}_{t \in \mathbb{N}_0}}        				        % Cadena de Markov
  \newcommand{\pnm}{\{X_t\}_{t \in [0, \infty[}}            				        % Proceso de nacimiento y muerte

% Para escalar matemáticas:
  \newcommand{\scalemath}[2]{\scalebox{#1}{\mbox{\ensuremath{\displaystyle #2}}}}

% TEOREMAS Y ENTORNOS ASOCIADOS

  % \newtheorem{theorem}{Theorem}[chapter]
  \newtheorem*{teorema*}{Teorema}
  \newtheorem{teorema}{Teorema}[chapter]
  \newtheorem{proposicion}[teorema]{Proposición}
  \newtheorem{lema}[teorema]{Lema}
  \newtheorem{corolario}[teorema]{Corolario}

    \theoremstyle{definition}
  \newtheorem{definicion}[teorema]{Definición}
  \newtheorem{ejemplo}[teorema]{Ejemplo}

    \theoremstyle{remark}
  \newtheorem{observacion}[teorema]{Observación}
